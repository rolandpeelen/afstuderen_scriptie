

%----------------------------------------------------------------------------------------
%  Introduction
%----------------------------------------------------------------------------------------
\chapter{Methodology}
\section{Introduction}
Developing a product configurator is by any means no easy feat. Building it using 3d technology may be even harder. When doing a small documentary analysis to get and set a proper theoretical framework, it became abundantly clear that there is a lot of research done into the computer science behind the proposed technology (\cite{openGLsite}, \cite{microservices}, \cite{heteregoneousComputingTechniques}). Next to that, loads of research has been done into the User Experience side of (web)applications (\cite{nielsonNormanReports}). Unfortunately, the main question that needs to be answered is quite specific and at this intersection of UX and Computer Science, there is little to no prior research. This means that some of the questions outlined below will need to be answered in an experimental way. There are a couple that can only be answered using this experimental type of setting. However, there are also some questions that could be answered using interviews or a documentary analysis. But with the need of actually building a prototype, going that way would be harder than to device an experiment using the prototype.
Basically, all questions that have to do with the development cycle, can be answered using this experimental type of research. Questions regarding the UX mostly will follow a combination of multiple methods (as we need to define certain standards). The start of this section will be a deep dive into the methodology behind building the prototype, followed by the ways of researching the development and user experience research objectives.

%----------------------------------------------------------------------------------------
%   Experimental Prototype
%----------------------------------------------------------------------------------------
\section{Experimental Prototype}
\subsection{Introduction}
To keep the prototype as simple as possible, an analysis of the requirements is made. After which a development stack shall be chosen and the development will be started. As a rough example, the example used in the theoretical framework shall be used. Most of the workflow in the build of this prototype shall be inherited from Pepprs current Software Workflow.

\say{
A chair manufacturer wants a product configurator for one of their most popular chairs. It has 25 different colour options and has 4 different subframes. Two of the subframes are steel and can be either black or plain. The other two have wooden elements and have four wood colour options.
}

\subsection{Requirements}

\subsection{Functional Requirements}
Looking at the outline above, there are basically two things the configurator needs to do; switch models (for frames) and switch colours (for both frames and chair itself). This should (logically) all be wrapped in a user interface where the user can switch the colours and models with their cursors. Some other things, inherit to product configurators are needed as well. To map these requirements in a way Peppr would do, these would amount to the following;

\begin{itemize}
	\item As an end-user, I want a user friendly way to change the colour of the chair
	\item As an end-user, I want a user friendly way to change the model of the frame
	\item As an end-user, I want to see my newly build configurator in enough ways to make a proper assertion as to wether or not I would buy this object
\end{itemize}

Dissecting these user-stories, some conclusions can be made as to what the front-end application of the system should do.
\begin{itemize}
	\item As a front-end, I need to be able to load 3d models into my scene
	\item As a front-end, I need a way to switch 3d models
	\item As a front-end, I need to show the 3d model realistic enough so the end-user can make a buy / do not buy choice
	\item As a front-end, I need to have a way for the user to navigate the scene and see the model from several points-of-view
	\item As a front-end, I need a way to know what colours the chair might have, so I can show the user his / her options
	\item As a front-end, I need a way to keep track of what models are in my scene, so I can apply the colour to the right object
	\item As a front-end, I need a way to keep track of the current configuration, so switching between different configurator options will keep the current configuration
\end{itemize}

From here, some user-stories for the back-end can be made.
\begin{itemize}
	\item As a back-end, I need to be able to supply 3d models to the front-end
	\item As a back-end, I need to show the front-end what options are available (both for models, as well as colours)
	\item As a back-end, I need to supply multiple 3d models if any configuration requires more than one model for a subset
\end{itemize}

While always difficult to estimate the exact requirements upfront, building a product configurator prototype using this short set of functional requirements should be sufficient to answer the research questions.

\subsection{Technical Requirements}

\subsection{Technology Stack}
\subsection{Graphics Cycle}
\subsection{Development}

%----------------------------------------------------------------------------------------
%   Development
%----------------------------------------------------------------------------------------
\section{Development}

%----------------------------------------------------------------------------------------
%   Is it possible to eliminate all issues (time and automation) with regards to rendering?
%----------------------------------------------------------------------------------------
\subsection{Is it possible to eliminate all issues (time and automation) with regards to rendering?}

%----------------------------------------------------------------------------------------
%   What is the cost differences for development?
%----------------------------------------------------------------------------------------
\subsection{What is the cost differences for development?}
% Estimate time for both projects
% Find 3d pricing
% Find developer pricing for CMS
% Find developer pricing for custom
% Check server costs
% Check CMS average upgrade cyclus
% Check NPM package average upgrade cyclus
% Check main NPM package upgrade cyclus
% Initial, Server costs, Upgrade costs

%----------------------------------------------------------------------------------------
%   What are the differences in Workflows
%----------------------------------------------------------------------------------------
\subsection{What are the differences in Workflows?}

%----------------------------------------------------------------------------------------
%   How flexible is the end result compared to previous generation configurators
%----------------------------------------------------------------------------------------
\subsection{How flexible is the end result compared to previous generation configurators}
% Estimate time in old way
% Find 3 different product configurators
% See how far we get re-building them using the CMS
% Estimate time in new way

%----------------------------------------------------------------------------------------
%   How easy is it so maintain the application compared to existing configurators?
%----------------------------------------------------------------------------------------
\subsection{How easy is it so maintain the application compared to existing configurators?}
% Interview Frederick with regards to maintaining the application / costs (what is it like owning a setup like that)
% --> from "What is the cost difference for development?"
% Use CMS average upgrade cyclus
% Use NPM package average upgrade cyclus
% Use main NPM package upgrade cyclus

%----------------------------------------------------------------------------------------
%   How easy is it to add new options to the configurator compared to the existing configurators?
%----------------------------------------------------------------------------------------
\subsection{How easy is it to add new options to the configurator compared to the existing configurators?}
% Estimate time in old way
% Add new options to configurator
% Estimate time in new way

%----------------------------------------------------------------------------------------
%   User Experience
%----------------------------------------------------------------------------------------
\section{User Experience}
Apart from the development cycle, the user experience is a huge part of whether using these types of configurators is viable or not. Even if the development is way cheaper, if users cannot use the application, problems will arise and the old method may prove to be a better option.

%----------------------------------------------------------------------------------------
%   Is the technology compatible with the users browsing preferences?
%----------------------------------------------------------------------------------------
\subsection{Is the technology compatible with the users browsing preferences?}
% Scan 'caniuse.com' for theoretical problems
% Setup domain with browserify and see how it renders

%----------------------------------------------------------------------------------------
%   Is the new technology small enough?
%----------------------------------------------------------------------------------------
\subsection{Is the new technology small enough?}
% Record network activity for three other product configurators that have a comparable setup
% Record network activity for OpenGL version
% Find average site size
% Find psychology behind it

%----------------------------------------------------------------------------------------
%   Is the new technology small enough?
%----------------------------------------------------------------------------------------
\subsection{Is the new technology quick enough?}
% Old -- Record / time switching between 'models'
% Old -- Record / time switching between 'materials'
% New -- Record / time switching between 'models'
% New -- Record / time switching between 'materials'
% Use data from bandwidth to calculate average loading time per network type (2g, 3g, 4g, etc.)