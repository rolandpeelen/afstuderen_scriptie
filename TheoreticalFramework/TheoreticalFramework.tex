\newpage
\section{Theoretical Framework}
\subsection{Introduction}

To set our perspective, a couple of subjects need to be touched to set a proper perspective. Starting with the underlying technology that might make these kind of projects succeed; WebGL and OpenGL. After that there are numerous market specific parts we need to address, starting with the current state of product configurators, Gartners Hypecycle (and wether or not he has spoken about such hypes) and finally, the method we'll be using to analyse all these domains, on which we shall base our conclusion.
Finally, we shall state an hypothesis that can be tested in this research.

\subsection{OpenGL \& WebGL}
\say{
WebGL is an application programming interface (API) for advanced 3d graphics on the web. It is based on OpenGL ES 2.0, and provides similar rendering functionality, but in an HTML and JavaScript context. The rendering surface that is used for WebGL is the HTML5 canvas element, which was originally introduced by Apple in the WebKit open-source browser engine. The reason for introducing the HTML5 canvas was to be able to render 2D graphics in applications such as Dashboard widgets and in the Safari browser on the Apple Mac OS X operating system.
Based on the canvas element, Vladimir Vukicevic at Mozilla started experimenting with 3d graphics for the canvas element. He called the initial prototype canvas 3d. In 2009 the Khronos Group started a new WebGL working group, which now consists of several major browser vendors, including Apple, Google, Mozilla, and Opera. The Khronos Group is a non-profit industry consortium that creates open standards and royalty-free APIs. It was founded in January 200 and is behind a number of other APIs and technologies such as OpenGL ES for 3d graphics for embedded devices, OpenCL for por parallel programming, OpenVG for low-level acceleration of vector graphics and OpenMAX for accelerated multimedia components. Since 2006 the Khronos Group has aso controlled and promoted OpenGL, which is a 3d graphics API for desktops.
The final WebGL 1.0 specification was frozen in March 2011, and WebGL support is implemented in several browsers, including Google Chrome, Mozilla Firefox, and (at the time of this writing) in the development releases of Safari and Opera.
}
\cite{Andreas Anyuru}
\newline

At the time of Andreas' writing, WebGL was very much in its early development stage. Anno 2015 though, WebGL has found its way into mainstream browsers with Internet Explorer and Android support lacking for versions earlier than its absolute latest \cite{Can I Use}, with an adoption rate of nearly 81\%. The most interesting part of using WebGL is the "low-level acceleration". Which means that on the stack of available hardware and software to render graphics, WebGL and OpenGL are on the lower ones (or closer to hardware level). This makes using WebGL and OpenGL insanely fast when compared to software / browser acceleration, which is very high-level.
The main bottleneck for using said low-level acceleration, is that when there is no low-level graphics processing unit (GPU) available, the computing needs to be done with high-level software, which is ultimately directed (through several levels of architecture) to the CPU of the device and some tasks are great for CPU calculations, however, some are not.
\newline

\say{
Both CPU and GPU possess distinct architectural features. Modern multicore CPUs use up to a few tens of cores, which are typically out-of-order, multi-instruction issue cores. Also, CPU cores run at high-frequency and use large sized caches to minimize the latency of a single-thread. Clearly, CPUs are suited for latency-critical applications. In contrast, GPUs use much larger number of cores, which are in-order cores that share their con-trol unit. Also, GPU cores use lower frequency,and smaller sized caches[Mittal 2014a].Thus, GPUs are suited for throughput-critical applications.
}
\cite{Heterogeneous Computing Techniques}
\newline

What is stated by Mittal and Vetter is the very basic, low-level difference between a CPU and GPU. The CPU is designed to handle any calculation thrown at it, while the GPU is designed to be more special purpose.
The importance of the CPU vs GPU is that WebGL makes use of the low-level acceleration of the GPU and more importantly, most modern smartphones these days have a GPU, making a potential adoption rate for the configuration platform much higher as end-users might well be able to use them on the go as well.

\subsection{Current State of Product Configurators}

\newpage