\newpage
\chapter{Current state}
\section{Introduction}

A couple of subjects need to be touched to set a proper perspective. Starting with the underlying technology that might make these kind of projects succeed; WebGL and OpenGL. After that there are numerous market specific issues we need to address, starting with the current state of product configurators, Gartners Hypecycle (and wether or not he has spoken about such hypes) and finally.

\section{Service Domain}

\section{Technical Domain}
Part of why this thesis is written, is the way these projects are handled currently. Peppr has done these projects in the past and states that the usual way of producing these consists of two sub projects; 3d renders, and software. In the outline below follows a small description of how the process is currently done. After that an introduction on OpenGL and WebGL.
\newline
\say{
A chair manufacturer wants a product configurator for one of their most popular chairs. It can be delivered in 25 different colours and has 4 different subframes. Two of the subframes are fully made from steel and can be delivered in either black or plain steel, the other two have wooden elements and have four different wood colour options.
}

\subsection{3d Department}
\subsubsection{Image Planning}
Production configurators often have semi-exponential expanding set of properties. Below a summary of the full option set split so we can start calculating how many images will be necessary.
\begin{itemize}
	\item 25 colours (\( \alpha \))
	\item 2 steel frames (\( \beta \))
	\item 2 frame colours (\( \gamma \))
	\item 2 wood frames (\( \delta \))
	\item 4 wood colours (\( \epsilon \))
\end{itemize}

To calculate the full amount of options ($x$), we can use the following formula:
 
\[ f(x) = (\alpha \cdot \beta \cdot \gamma) + (\alpha \cdot \delta \cdot \beta \cdot \epsilon)\]
\[ f(x) = (25 \cdot 2 \cdot 2) + (25 \cdot 2 \cdot 2 \cdot 4)\]
\[ f(x) = 500\]

Going with just 1 colour more adds 20 extra renders. So while the option set might not seem like much, adding extra options is costly. In some cases, you might be able to get around it using layers.
Splitting up this case into a seat and frame layer will get us the following formula:

\[ f(x) = (\alpha) + (\cdot \beta \cdot \gamma) + (\cdot \delta \cdot \beta \cdot \epsilon)\]
\[ f(x) = (25) + (2 \cdot 2) + (2 \cdot 2 \cdot 4)\]
\[ f(x) = 116\]

This decreases the amount of renders by around 65\% - 70\%. Unfortunately, this is not possible in all cases (due to the fact that a layer might be both in front and to the back of something simultaneously, making it extremely difficult to 'mask'), and creating new layers does add complexity, not only in the following processes but also in the software department, having to create either an API that serves those layers as one image or a front-end that can layer multiple images in a correct way. In the lather case, the user will directly notice this option as it adds at leaste one more http-request.

\subsubsection{Modeling}
Step two in the chain is the modeling proces whereby a virtual model is being created, either from scratch, or by 3d scanning real-life objects. If the choice to go for layers in the renders have been made, the modeler must make sure that there are clear creases where the cuts of those layers go so there is no overlapping geometry.
\subsubsection{Lighting \& Shading}
In the lighting department, the model is being put into a suitable environment (mostly studio like setups), where it is lit and shaded (process of creating a life-like material) to perfection. The lighter must make sure that when there are layers involved, there is no unwanted shadow casting because when hiding certain layers for rendering.
\subsubsection{Render}
This is where things get together and the calculation from 3d model to actual image start. Depending on the configuration and setup, one renders the entire sequence, if < 100 renders and only one point of view for instance, one might swap out the model at certain frames. If one needs to render 360's, rendering one file at at time is better suitable.
\subsubsection{Post Production}
Every 3d model needs a bit of post production to make it look better. Also, if the model was split out into different layers but these were rendered as masks (an option to overcome overlapping images), this is where they would be split out into the different layers.
\subsubsection{Compression}
Using images of the web, especially on @2x or @3x resolution devices, mostly phones and tablets which get their date through mobile networks, compression is extremely important. Laying a 50mb burden upon the user when opening a site is not a good idea.

\subsection{Software}
\subsubsection{Requirements}
Starting a software project should start with working out both functional and technical requirements. These will form a base to develop an API and front-end in a way that does not (should not) surprise the makers.
\subsubsection{To CMS or not to CMS}
This is a question that is tricky to answer. A content management system has basic functionality built in (user management, file handling, basic front-end), but it does require to work in the way that system is meant to be worked in. The other option, going with a from the ground up written system, will be much leaner and quicker when deployed, but will not be as mature as a popular CMS, which might result in a buggy experience for the end user.
\subsubsection{API Planning}
If 
\subsubsection{UX Development}
\subsubsection{Front-End Development}

\section{Operational Domain}

\section{Financial Domain}

\newpage
