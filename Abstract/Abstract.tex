\newpage
\section{Abstract}

Peppr is a company that specialises itself in building photo-realistic visualisations. Last year, Peppr was asked to build a product configurator for a company that makes bespoke tailor made shirts so their customers would be able to design their own shirts with custom colours, sleeves, buttons etc. 

The current way of building said configurator, is to split up the product into different layers that resemble the options and create images for every configuration. In Peppr's case, there where 25 colours, 2 perspectives, 7 collars, 6 sleeves and 3 base shirts. This left Peppr with a sum of 6300 different images that had to be made, and if the client wanted to add another color, they would have to build another set of 252 images. This is a very timely and costly venture.

Thus, Peppr concluded the usual way of doing these types of projects is suboptimal, and started looking for an alternative. That's where this project comes to play. March 2011 (https://en.wikipedia.org/wiki/WebGL) was the first release of WebGL, an online OpenGL rendering system that directly renders to a computers hardware. This opened the web to a whole new way of using it. Ever since, the actual option rate has always been low because only the latest browsers would integrate the technology, anno 2015 however, the playing field has changed. With more and more browsers supporting this new type of technology, the timing might be perfect to start bringing it to the masses.

In this thesis I will try to find if a WebGL based product configurator, with at heart and easy-to-use Content Management System, is feasible as a business proposition.
\newpage