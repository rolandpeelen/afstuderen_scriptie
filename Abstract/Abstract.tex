\chapter*{Abstract}%

Peppr is a company that is specialized in building photo-realistic visualizations. Late 2014, Peppr built a product configurator for SlimFitted, a company that makes tailored shirts. Te customer wanted to built a platform on which their customers would be able to customize their own shirts 

Most web-based product configurators are built by splitting up the full productimage into different layers, (mostly) one for each different configurable component. In Peppr's case, the customer wanted 25 colors, 2 perspectives, 7 types of collars, 6 types of sleeves and 3 different plaids. This left Peppr with a sum of 6300 different combinations. Even with the smartly dividing the images, Peppr need to make a staggering amount of images.

Next to the massive amount of images, there is also a problem that needs to be solved with regards to software. The end-user needs to be able to view and edit the shirt configuration in a user-friendly way. As of yet, most configurators have vastly different option sets (with different amounts of layers), components and style. This makes building a universal software application to handle all product configurators (in current manner) extremely difficult. 

That is where this thesis comes to play. March 2011 was the first release of WebGL (https://en.wikipedia.org/wiki/WebGL), an implementation of OpenGL technology for the web. WebGL technology renders directly from the video processor and it opens up the web to a whole new way of using actual 3d models. Being able to use the 3d models instead of the vast amount of images would make the configurators easier to build, and thus, less expensive to make and maintain. Unfortunately, the actual adoption rate of WebGL has always been low because only the latest browsers would integrate the technology. Anno 2016 though, the playing field has changed. With more-and-more browsers supporting this new type of technology, the timing might be perfect to bring it to the masses.

In this thesis, we will explore the possibilities of said configurator and try to find if this technology is ripe for production.