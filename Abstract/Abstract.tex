\chapter*{Abstract}%

Peppr is a company that is specialized in building photo-realistic visualizations. Late 2014, Peppr built a product configurator for SlimFitted, a company that builds tailored shirts. They wanted their customers to be able to design their own shirts. 

Current product configurators for the web are built by splitting up the product into different layers. Every layer consists of a pack of images. Which, in Peppr's case, were 25 colors, 2 perspectives, 7 collars, 6 sleeves and 3 base shirts. This left Peppr with a sum of 6300 different layers (and thus, images) and when the client wants to add another color, they would have to build another set of 252 images. This is a timely and costly venture.

Next to the render aspect, there is also the question of software, as these images need to be viewed and configured by the customer. As such, Peppr concluded the usual way of doing these types of projects is suboptimal and started looking for an alternative. That is where this thesis comes to play. March 2011 was the first release of WebGL (https://en.wikipedia.org/wiki/WebGL), an implementation of OpenGL technology for the web. Because WebGL renders directly from the video processor, it opens up the web to a whole new way of using 3d. The actual adoption rate has always been low because only the latest browsers would integrate the technology. Anno 2015 though, the playing field has changed. With more-and-more browsers supporting this new type of technology, the timing might be perfect to bring it to the masses.

This does bring new possibilities and challenges. For one, if Peppr succeeds in its mission, WebGL configurators might be more flexible, allowing for a Content Management System (CMS) like setup. This in turn would mean that Peppr could serve this as a Software As A Service (SAAS) business model. This would however mean, that Peppr might need to rebrand itself to be a software company, or set the service up as a separate entity. Many questions arise when thinking about the possibilities, risks and challenges.

In this thesis I will try to find if a WebGL based product configurator, is viable in terms of service, technical, operational and financial aspects.