\chapter*{Abstract}%

As a company that specialises in high-end photorealistic imagery, Peppr is always on the lookout for new technologies to achieve that. As was the case with their product configurators. They built one using state-of-the-art technology but wanted something better as it was not flexible enough and adding options was cumbersome. In their own research they found a way to show realtime 3d models in browser using WebGL and wanted to know whether or not it would be a viable alternative to pre-rendering all options of the configurator. \newline
After dissecting the workflow Peppr used to build the configurator, research objective were set. Both in development and user experience there were key-points that would determine either the success or failure of the technology. From their current workflow, rendering and automating made every single project difficult. This was a point that most definitely needed research. The differences in workflow and accompanying costs would be others.\newline
The next step was to look from another perspective, that of the user. The User Experience is an important part of a configurator as it partly determines wether users will continue using the configurator to eventually buy something or end up confused and annoyed wanting to buy at competitors. Three things were tested in this thesis: The amount of data the user needs to download to use the configurator (and how quick that loads), how the WebGL configurator performs on speed tests and, if the WebGL technology is compatible with the current browser technology. \newline
Finding literature or previous research at this intersection of cutting edge technology and user experience proved extremely difficult. This resulted in building a prototype of a WebGL configurator. The prototype featured an AngularJS and ThreeJS front-end that loads its data from a JSON style Angular constant. With the prototype in place, testing commenced. \newline
Looking at the research objectives that came from the development cycle, there was a clear victor. A saving of 8.6\% in the creation of the application was just the beginning. In their current workflow, adding a component would add an estimate of 5 hours, while the WebGL version would take just 15 minutes. When adding an extra component, the change is even bigger. In their current workflow, it would take roughly 27 hours, while WebGL version would take just 1 hour and 40 minutes. \newline
Some interesting results were found from the UX perspective too. Firstly, that while the 3d models are relatively big when compared to images, the page loads are actually less than the previous configurator Peppr built. This was partly due to the fact that Peppr chose to preload all images for one configuration in their current workflow and partly due to the fact images were used and they can only be compressed so much before losing quality. Secondly, the application was quite fast and responsive, rendering at an average 19fps on a two year old iPhone 6. Lastly, the application rendered and was usable on 12 of the latest devices and browsers. \newline
All-in-all, while there are still some improvements that could be made on the prototype, the conclusion drawn from all objectives is that even without those optimizations, the technology saves time, is just as small or even smaller, works quick enough, and is compatible on all modern browsers. As such, a WebGL product configurator is most definitely a viable alternative to the current workflow at Peppr.