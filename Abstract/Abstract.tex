\chapter*{Abstract}%

Peppr is a company that specialises in building photo-realistic visualisations. Late 2014, Peppr built a product configurator for SlimFitted, a company that makes tailored shirts. They wanted a platform on which their customers would be able to order customized shirts.

The customer requested 25 colors, 2 perspectives, 7 types of collars, 6 types of sleeves and 3 different plaids. This would amount to 6300 possible shirt combinations. Creating all the combinations is massive undertaking, which it why most web-based product configurators are built by splitting the final image into several configurable components. But even with the smart division of layers, Peppr still needed to render a staggering amount of images.

Apart from the images, Peppr encountered a problem that needed to be solved with regards to software. For every configurator, the end-user needs to be able to view and edit a configuration in a user-friendly way. Unfortunately, this means something different for all configurators. Combined with branding, creating a universal and more importantly, re-usable configurator is extremely difficult.

That is where this thesis comes to play. Peppr stumbled upon a technology called OpenGL which was released in March 2011 (\cite{OpenGL Website}), an implementation of OpenGL technology for the web. WebGL technology renders directly from the video processor and it opens up the web to a whole new way of using actual 3d models. Being able to use the 3d models instead of the vast amount of images would make the configurators easier (less expensive) to build and maintain. Unfortunately, the actual adoption rate of WebGL has always been low because only the latest browsers would integrate the technology and there were hardware limitations on mobile phones. Anno though, the playing field has changed. With more-and-more browsers supporting this new type of technology, the timing might be perfect to bring it to the masses.

In this thesis, we will explore the possibilities of said configurator and try to find if this technology is ripe for production.