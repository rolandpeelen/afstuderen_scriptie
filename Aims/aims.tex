%----------------------------------------------------------------------------------------
%  Introduction
%----------------------------------------------------------------------------------------
\chapter{Aims}
\section{Introduction}
The main goal of this thesis is to find whether using WebGL is a viable option when building configurators. This objective has two sections, the development and the user experience of the application. Below is a brief summary of each section, ending with an aim for that specific property.
%----------------------------------------------------------------------------------------
%   Development
%----------------------------------------------------------------------------------------
\section{Development}
Developing a product configurator is by any means no easy feat. Building it using 3d technology may be even harder. The following three questions should be answered in this thesis.
​\begin{enumerate}
\item What is the development cost difference both in time and money?
\item Is the end-result flexible enough to be re-used for other clients?
\item How easy is it so maintain the application compared to existing configurators?
\item How easy is it to add new options to the configurator compared to the existing configurators?
\end{enumerate}

%----------------------------------------------------------------------------------------
%   User Experience
%----------------------------------------------------------------------------------------
\section{User Experience}
Apart from the development cycle, the user experience is a huge part of whether using these types of configurators is viable or not. Even if the development is way cheaper, if users cannot use the application, problems will arise and the old method may prove to be a better option.\begin{enumerate}
\item Is the technology compatible with the users browsing preferences?
\item Is the technology small enough?
\item Is the technology quick enough?
\end{enumerate}