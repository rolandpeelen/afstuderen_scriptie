%----------------------------------------------------------------------------------------
%  Introduction
%----------------------------------------------------------------------------------------
\chapter{Research Objectives}
\section{Introduction}
Peppr wants to know wether or not a WebGL product configurator that directly shows 3d models instead of images is a viable alternative to their current product configurators.. Not only in development but also in end user experience. To answer this, a deep dive into their current workflow and possible new technology has been done. In that dive, numerous questions arised that should help answer the main research objective. They are outlined below. Most of the objectives are open questions for which it is hard to specify a set of requirements for when that goal is met. They way these questions are answered is described in gread detail in chapter \ref{chapter:methodology}.
%----------------------------------------------------------------------------------------
%   Development
%----------------------------------------------------------------------------------------
\section{Development}
Developing a product configurator is challenging. Building it with realtime 3d technology may be even harder. The following three questions should be answered in this thesis.
​\begin{enumerate}
\item {Is it possible to eliminate all issues (time and automation) with regards to rendering?}
\item{What are the cost differences for development?}
\item {What are the differences in Workflows}
\end{enumerate}

%----------------------------------------------------------------------------------------
%   User Experience
%----------------------------------------------------------------------------------------
\section{User Experience}
Apart from the actual development of the configurator, the user experience is a huge part of whether using WebGL configurators is viable or not. Even if the development is cheaper, when users cannot use the application, problems will arise and the old method may prove to be a better option.\begin{enumerate}
\item {Is the technology new small enough?}
\item {Is the technology new quick enough?}
\item {Is the technology compatible with the user's browsing preferences?}
\end{enumerate}