%----------------------------------------------------------------------------------------
%  Introduction
%----------------------------------------------------------------------------------------
\chapter{Aims}
\section{Introduction}
Peppr wants to know wether or not a WebGL product configurator that directly shows 3d models instead of images is a viable alternative to their current workflow. Both in development, as in end user experience. To answer this, a deep dive into their current workflow and possible new technology has been done. In that deep-dive, numerous questions arised that should help answer the main research objective. They are outlined below. With every goal, a small description with a way to determine wether or not the goal has been met. In the next section, a more elaborate description will follow, combined with a method of research.
%----------------------------------------------------------------------------------------
%   Development
%----------------------------------------------------------------------------------------
\section{Development}
Developing a product configurator is by any measure no easy feat. Building it with realtime 3d technology may be even harder. The following three questions should be answered in this thesis.
​\begin{enumerate}
\item {Is it possible to eliminate all issues (time and automation) with regards to rendering?}
\item{What are the cost differences for development?}
\item {What are the differences in Workflows}
\end{enumerate}

%----------------------------------------------------------------------------------------
%   User Experience
%----------------------------------------------------------------------------------------
\section{User Experience}
Apart from the development cycle, the user experience is a huge part of whether using these types of configurators is viable or not. Even if the development is way cheaper, if users cannot use the application, problems will arise and the old method may prove to be a better option.\begin{enumerate}
\item {Is the technology new small enough?}
\item {Is the technology new quick enough?}
\item {Is the technology compatible with the users browsing preferences?}
\end{enumerate}