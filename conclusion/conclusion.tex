\chapter{Conclusion}

As a company that specialises in high-end photorealistic imagery, Peppr is always on the lookout for new technologies. As was the case with their product configurators. They built one using then state-of-the-art technology but wanted something better. In their own research they found WebGL and wanted to know whether or not it would be a viable alternative to pre-rendering all options of the configurator. \newline
After starting out dissecting the workflow Peppr used to build their configurator, research objective were set. Both in development and user experience there were key-points that would determine either the success or failure of the technology. The first being the question if it would be possible to eliminate all issues with regards to rendering and automation. After looking at both their current workflow and the one used to build an experimental prototype, the results showed that some of the steps previously done in the 3d department were now shifted to the software side. The steps that shifted were the steps that were associated with the issues Peppr was having with rendering and automation. While there is still the possibility to pre-render the models and bake the lighting information onto the model so that it renders a bit quicker in realtime, for the purpose of the proposed configurator (and as shown later with the FPS tests), this was not necessary. As such, the only conclusion that can be made with regards to these issues is that they are eliminated in the WebGL version of the configurator.\newline
The next research objective was to look at the cost differences of the two configurators and the differences in workflow. The results showed a movement of time from the 3d department to the software department (as to be expected), but an overall decrease of 8.6\% in costs for the creation of the initial configurator. While maintenance on the application had to be done more often than with the current CMS setup, a move to a subscription model and centralized API / configurator could eliminate this. The WebGL configurator is more reliant on software for its configuration than the CMS setup, this means it is more flexible and a setup like the aforementioned could be possible. This would reduce maintenance costs. Apart from those, the cost difference was calculated for adding a new colour and component to the configurator in both cases. Estimated for the colour in the current workflow was 5 hours, while 15 minutes were estimated in the WebGL workflow. When adding an extra component, the change is even bigger. In their current workflow, it would take roughly 27 hours, while WebGL version would take just 1 hour and 40 minutes. All-in-all, the conclusion with regards to this objective is that the WebGL workflow is cheaper and more flexible than the current workflow.\newline

After looking at all objectives, the perspective changed and a look was taken from the user's point-of-view. Three things were looked at, the first one being whether or not the WebGL 3d models were not to big to use on the web. The results were quite surprising. The 3d models themselves are relatively big when compared to images, but the initial page load was smaller. This was partly due too the fact that Peppr chose to preload all images for one configuration in their current workflow and partly due too images being used and images can only be compressed so much before losing quality. When starting to use the configurator however, some changes were slightly in favour of Peppr's current workflow. As soon as actual models started to be swapped, the amount of megabytes being downloaded went up rather quickly. The main reason for this was the lack of caching and compression of 3d models. This is something that was not touched in this thesis and remains to be researched. However, the conclusion remains that the load of the configurator in the most extreme case for the WebGL version was 10mb and more than three times as much (33.9mb) for the current configurator. With some optimization, this may be even more. As such, this too is in favour of the WebGL configurator. \newline
After looking at the page load in terms of size, the same thing was done for it in terms of time. There were no real surprises other than the user spending more than 6 minutes of just loading when opening the current configurator and making 5 colour changes. The second thing tested with regards to the speed of the configurator, was the amount of frames per second while viewing and interacting with the WebGL configurator. The slowest device (an iPhone 6) averaged out on 19fps with the slowest when interacting at 17fps. A Macbook averaged out at 27fps and a Mac Pro displayed everything at 60fps. While the slowest device did not appear completely smooth, it sill displayed a framerate high enough for humans to be perceived as motion (\cite{frameRate}). Apart from that, little to no optimization on the models or scene has been done. It may well be that on a lower polygon model or a differently lit scene, the framerate would be much higher. All-in-all, even the 19fps on a two year old device seems acceptable. As such, the conclusion here is that the speed of the WebGL configurator is definitely good enough for a production version. \newline
The last research question that had to be answered was one of compatibility. Specifically, if modern browsers have implemented the technology to handle the WebGL configurator. While only the latest browsers were tested and one could argue users do not always have this version, most modern browsers upgrade automatically and the amount of people using a more modern browser is growing every day. Next to that, the configurator as proposed here is not implemented yet, the browsers of today are the history of tomorrow. This means the assumption may be made that when the configurator actually hits production, there are even newer browsers versions. The data for the current browsers showed that every single one supported the rendering of the chair, the switching of frames, and the switching of colours.\newline

All-in-all, while there are still some improvements that could be made on the prototype, the conclusion drawn from all objectives is that even without those optimizations, the technology saves time, is just as small or even smaller, works quick enough, and is compatible on all modern browsers. As such, a WebGL product configurator is most definitely a viable alternative to the current workflow at Peppr.

The prototype can be found here: \url{http://peppr-configurator.herokuapp.com/}