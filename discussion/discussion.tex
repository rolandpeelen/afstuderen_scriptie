\chapter{Discussion}
%----------------------------------------------------------------------------------------
%   Development
%----------------------------------------------------------------------------------------
\section{Development}
Developing a product configurator is by any measure no easy feat. Building it with realtime 3d technology may be even harder. The following three questions should be answered in this thesis.

\subsection {Is it possible to eliminate all issues (time and automation) with regards to rendering?}
\label{discuss:developmentElimination}
WebGL renders the loaded models in realtime using software specified shaders. There was no rendering needed. This meant the ability to eliminate all issues was there. There are some other points to consider though, which shall be discussed in detail in section 

This question was answered by combining the experience taken from building the prototype and the information given by Peppr. They helped outlining the current workflow so a good comparison between the current process and the hypothesized new process could be made.


\subsection{What are the cost differences for development?}
\subsection {What are the differences in Workflows}

%----------------------------------------------------------------------------------------
%   User Experience
%----------------------------------------------------------------------------------------
\section{User Experience}
Apart from the development cycle, the user experience is a huge part of whether using these types of configurators is viable or not. Even if the development is way cheaper, if users cannot use the application, problems will arise and the old method may prove to be a better option.

\subsection {Is the technology new small enough?}
\subsection {Is the technology new quick enough?}
\subsection {Is the technology compatible with the user's browsing preferences?}