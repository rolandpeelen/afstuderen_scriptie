%----------------------------------------------------------------------------------------
%   Introduction
%----------------------------------------------------------------------------------------
\chapter{Introduction}

Peppr is a company that is specialised in building visualisations and animations. The company was founded in 2007 and from the start their focus was on creating high-end, photorealistic imagery. They try to stay up-to-date with new technologies and are always up for a challenge. The company consists of 4 people that bind themselves using a partnership agreement.

In 2014, Peppr was asked to build a product configurator for one of their customers (SlimFitted, a company that makes tailored shirts). They wanted a platform on which their customers would be able to order tailor made shirts. The customer requested the configurator would have options for 25 colors, 2 perspectives, 7 types of collars, 6 types of sleeves and 3 different plaids. The amount of possible shirt combinations was a staggering 3.150 (25 * 7 * 6 * 3) and all of those had to be rendered from two perspectives. Creating the set of images that would make this a possibility was massive. Smartly dividing the images into parts and only changing the requested image would make things a bit easier, but the overall image-set would still be huge.

Apart from the actual images needed, the end-user needs to be able to view and edit a configuration in a simple way. This means a piece of software had to be built to show the configuration to the user. In an ideal scenario, Peppr builds a piece of software that can be re-used. Unfortunately, every customer will need their own branding and every configurator may have slightly different controls. Another difficulty is the image-set. It was huge to begin with, so Peppr split it up into several parts. This meant they could change the collar while retaining the rest of the image, for example. This was a tricky process and adding the complexity to make it re-usable would have been a massive undertaking in both and money, resulting in a configurator that is more complex and prone to bugs because of the added complexity.

Peppr feels there must be a better way to serve their customers in the future, offering their customers a cheaper, more maintainable configurator. Current technologies allow the viewing of 3d models directly in the browser. In 2015, Peppr got a second request to build a configurator and started researching the possibilities. When they did, they encountered a piece of technology called OpenGL (/ WebGL) and a javascript based API to use it on the web called "three.js". This type of technology makes for a 3d model that can be shown directly in a webbrowser (hardware accelerated) instead of pre-rendered images. It may well be the next big thing when it comes to configurators.

The second customer backed out, but Peppr felt the need to explore the possibilities of WebGL nonetheless. To put it in concrete words: Is a WebGL product configurator that directly shows 3d models instead of images a viable alternative to their current workflow? Both in development, as well as end user experience.





