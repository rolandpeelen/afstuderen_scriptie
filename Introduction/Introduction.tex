%----------------------------------------------------------------------------------------
%   Introduction
%----------------------------------------------------------------------------------------
\chapter{Introduction}

Peppr is a company that is specialised in building visualisations and animations. They were founded in 2007 and from the start, focussed on creating high-end, photorealistic imagery. They try to stay up-to-date with new technologies and are always up for a challenge. The company consists of 4 people that bind themselves using a partnership agreement.

In 2014, Peppr was asked to build a product configurator for one of their customers (SlimFitted, a company that makes tailored shirts). They wanted a platform on which their customers would be able to order customised shirts. The customer requested the configurator had 25 colors, 2 perspectives, 7 types of collars, 6 types of sleeves and 3 different plaids. This would amount to 6300 possible shirt combinations. Even with a smart division in layers, making all those combinations is a massive undertaking.

Apart from the actual images needed, the end-user needs to be able to view and edit a configuration in a simple way. This means a piece of software had to be built to show the configuration to the user. Unfortunately, every customer wants their own branding and with the nature of layered images (and the technical difficulties they create), creating a universal and re-usable configurator is difficult.

Peppr feels there must be a better way to serve their customers, so they will end up with a cheaper, more maintainable configurator. Current technologies allow the viewing 3d models directly in the browser. In 2015, Peppr got a second request to built a configurator and started researching their possibilities. When they did, they stumbled upon a piece of technology called OpenGL (/ WebGL) and a javascript based binding to use it on the web called "three.js". This type of technology means that a 3d model can be shown directly in browser (hardware accelerated) and it could be the next big thing when it comes to configurators.

The second clients order fell through, but Peppr felt the need to research this further and wants to know this. Is a WebGL product configurator that directly shows 3d models instead of images a viable alternative to their current workflow. Both in development, as in end user experience.





