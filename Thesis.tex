%%%%%%%%%%%%%%%%%%%%%%%%%%%%%%%%%%%%%%%%%
%
% Graduation Thesis RWJ Peelen
%
%%%%%%%%%%%%%%%%%%%%%%%%%%%%%%%%%%%%%%%%%
%----------------------------------------------------------------------------------------
%   PACKAGES AND OTHER DOCUMENT CONFIGURATIONS
%----------------------------------------------------------------------------------------

\documentclass[14pt]{article}
\usepackage[english]{babel}
\usepackage[utf8x]{inputenc}
\usepackage{amsmath}
\usepackage{graphicx}
\usepackage[colorinlistoftodos]{todonotes}

\begin{document}

%----------------------------------------------------------------------------------------
%   TitlePage
%----------------------------------------------------------------------------------------
\begin{titlepage}

\newcommand{\HRule}{\rule{\linewidth}{0.5mm}} % Defines a new command for the horizontal lines, change thickness here

\center % Center everything on the page
 
%----------------------------------------------------------------------------------------
%   HEADING SECTIONS
%----------------------------------------------------------------------------------------

\textsc{\Large Utrecht University for Applied Sciences}\\[1cm]
\textsc{\Large Digital Communication \& Media}\\[0.5cm]
\textsc{\large Faculty for Communication \& Journalism}\\[0.5cm]

%----------------------------------------------------------------------------------------
%   TITLE SECTION
%----------------------------------------------------------------------------------------

\HRule \\[0.4cm]
{ \huge \bfseries WebGL Product Configurators}\\[0.8cm]
\textsc{\large From Technical and Market Viability to Alpha}\\[0.5cm]
\HRule \\[1.5cm]
 
%----------------------------------------------------------------------------------------
%   AUTHOR SECTION
%----------------------------------------------------------------------------------------

\begin{minipage}{0.4\textwidth}
\begin{flushleft} \large
\emph{Author:}\\
RWJ \textsc{Peelen} % Your name
\end{flushleft}
\end{minipage}
~
\begin{minipage}{0.4\textwidth}
\begin{flushright} \large
\emph{Supervisor:} \\
Dhr. Kees \textsc{Winkel} % Supervisor's Name
\end{flushright}
\end{minipage}\\[2cm]

% If you don't want a supervisor, uncomment the two lines below and remove the section above
%\Large \emph{Author:}\\
%John \textsc{Smith}\\[3cm] % Your name

%----------------------------------------------------------------------------------------
%   DATE SECTION
%----------------------------------------------------------------------------------------

{\large Jan. 2016}\\[2cm] % Date, change the \today to a set date if you want to be precise

%----------------------------------------------------------------------------------------
%   LOGO SECTION
%----------------------------------------------------------------------------------------

% \includegraphics{logo.png}\\[1cm] % Include a department/university logo - this will require the graphicx package
 
%----------------------------------------------------------------------------------------

\vfill % Fill the rest of the page with whitespace

\end{titlepage}

%----------------------------------------------------------------------------------------
%   Abstract
%----------------------------------------------------------------------------------------
\newpage
\section{Abstract}

Peppr is a company that specialises itself in building photo-realistic visualisations. Last year, Peppr was asked to build a product configurator for a company that makes bespoke tailor made shirts so their customers would be able to design their own shirts with custom colours, sleeves, buttons etc. 

The current way of building said configurator, is to split up the product into different layers that resemble the options and create images for every configuration. In Peppr's case, there where 25 colours, 2 perspectives, 7 collars, 6 sleeves and 3 base shirts. This left Peppr with a sum of 6300 different images that had to be made, and if the client wanted to add another color, they would have to build another set of 252 images. This is a very timely and costly venture.

Thus, Peppr concluded the usual way of doing these types of projects is suboptimal, and started looking for an alternative. That's where this project comes to play. March 2011 (https://en.wikipedia.org/wiki/WebGL) was the first release of WebGL, an online OpenGL rendering system that directly renders to a computers hardware. This opened the web to a whole new way of using it. Ever since, the actual option rate has always been low because only the latest browsers would integrate the technology, anno 2015 however, the playing field has changed. With more and more browsers supporting this new type of technology, the timing might be perfect to start bringing it to the masses.

In this thesis I will try to find if a WebGL based product configurator, with at heart and easy-to-use Content Management System is feasible as a business proposition.
\newpage

%----------------------------------------------------------------------------------------
%   TOC
%----------------------------------------------------------------------------------------
\tableofcontents
\newpage


\section{Introduction}

Your introduction goes here! Some examples of commonly used commands and features are listed below, to help you get started.

If you have a question, please use the support box in the bottom right of the screen to get in touch. 

\section{Some \LaTeX{} Examples}
\label{sec:examples}

\subsection{Sections}

Use section and subsection commands to organize your document. \LaTeX{} handles all the formatting and numbering automatically. Use ref and label commands for cross-references.

\subsection{Comments}

Comments can be added to the margins of the document using the todo command, as shown in the example on the right. You can also add inline comments too:

\subsection{Tables and Figures}

Use the table and tabular commands for basic tables --- see Table~\ref{tab:widgets}, for example.

% Commands to include a figure:
%\begin{figure}
%\centering
%\includegraphics[width=0.5\textwidth]{frog.jpg}
%\caption{\label{fig:frog}This is a figure caption.}
%\end{figure}

\subsection{Mathematics}

\LaTeX{} is great at typesetting mathematics. Let $X_1, X_2, \ldots, X_n$ be a sequence of independent and identically distributed random variables with $\text{E}[X_i] = \mu$ and $\text{Var}[X_i] = \sigma^2 < \infty$, and let
$$S_n = \frac{X_1 + X_2 + \cdots + X_n}{n}
      = \frac{1}{n}\sum_{i}^{n} X_i$$
denote their mean. Then as $n$ approaches infinity, the random variables $\sqrt{n}(S_n - \mu)$ converge in distribution to a normal $\mathcal{N}(0, \sigma^2)$.

\subsection{Lists}

You can make lists with automatic numbering \dots

\begin{enumerate}
\item Like this,
\item and like this.
\end{enumerate}
\dots or bullet points \dots
\begin{itemize}
\item Like this,
\item and like this.
\end{itemize}

We hope you find write\LaTeX\ useful, and please let us know if you have any feedback using the help menu above.

\end{document}