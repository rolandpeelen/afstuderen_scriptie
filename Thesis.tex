%%%%%%%%%%%%%%%%%%%%%%%%%%%%%%%%%%%%%%%%%
%
% Graduation Thesis RWJ Peelen
%
%%%%%%%%%%%%%%%%%%%%%%%%%%%%%%%%%%%%%%%%%
%----------------------------------------------------------------------------------------
%   PACKAGES AND OTHER DOCUMENT CONFIGURATIONS
%----------------------------------------------------------------------------------------

\documentclass[14pt]{article}
\usepackage[english]{babel}
\usepackage[utf8x]{inputenc}
\usepackage{amsmath}
\usepackage{graphicx}
\usepackage[colorinlistoftodos]{todonotes}

\begin{document}

%----------------------------------------------------------------------------------------
%   TitlePage
%----------------------------------------------------------------------------------------
\begin{titlepage}

\newcommand{\HRule}{\rule{\linewidth}{0.5mm}} % Defines a new command for the horizontal lines, change thickness here

\center % Center everything on the page
 
%----------------------------------------------------------------------------------------
%   HEADING SECTIONS
%----------------------------------------------------------------------------------------

\textsc{\Large Utrecht University for Applied Sciences}\\[1cm]
\textsc{\Large Digital Communication \& Media}\\[0.5cm]
\textsc{\large Faculty for Communication \& Journalism}\\[0.5cm]

%----------------------------------------------------------------------------------------
%   TITLE SECTION
%----------------------------------------------------------------------------------------

\HRule \\[0.8cm]
{ \huge \bfseries WebGL Product Configurators}\\[1cm]
\textsc{\large Technical Viability and User Experience}\\[0.4cm]
\HRule \\[1.5cm]
 
%----------------------------------------------------------------------------------------
%   AUTHOR SECTION
%----------------------------------------------------------------------------------------

\begin{minipage}{0.4\textwidth}
\begin{flushleft} \large
\emph{Author:}\\
R.W.J. \textsc{Peelen} % Your name
\end{flushleft}
\end{minipage}
~
\begin{minipage}{0.4\textwidth}
\begin{flushright} \large
\emph{Supervisor:} \\
K. \textsc{Winkel} % Supervisor's Name
\end{flushright}
\end{minipage}\\[2cm]

% If you don't want a supervisor, uncomment the two lines below and remove the section above
%\Large \emph{Author:}\\
%John \textsc{Smith}\\[3cm] % Your name

%----------------------------------------------------------------------------------------
%   DATE SECTION
%----------------------------------------------------------------------------------------

{\large \today}\\[2cm] % Date, change the \today to a set date if you want to be precise

%----------------------------------------------------------------------------------------
%   LOGO SECTION
%----------------------------------------------------------------------------------------

% \includegraphics{logo.png}\\[1cm] % Include a department/university logo - this will require the graphicx package
 
%----------------------------------------------------------------------------------------

\vfill % Fill the rest of the page with whitespace

\end{titlepage}

%----------------------------------------------------------------------------------------
%   Abstract
%----------------------------------------------------------------------------------------
\newpage
\section{Abstract}

Peppr is a company that is specialized in building photo-realistic visualizations. Late 2014, Peppr built a product configurator for SlimFitted, a company that builds tailored shirts. They wanted their customers to be able to design their own shirts. 

Current product configurators for the web are built by splitting up the product into different layers. Every layer consists of a pack of images. In Peppr's case, there were 25 colors, 2 perspectives, 7 collars, 6 sleeves and 3 base shirts. This left Peppr with a sum of 6300 different layers (and thus, images) and when the client wants to add another color, they would have to build another set of 252 images. This is a timely and costly venture.

Peppr concluded the usual way of doing these types of projects is suboptimal and started looking for an alternative. That is where this thesis comes to play. March 2011 was the first release of WebGL (https://en.wikipedia.org/wiki/WebGL), an implementation of OpenGL technology for the web. Because WebGL renders directly from the video processor, it opens up the web to a whole new way of using 3d. The actual adoption rate has always been low because only the latest browsers would integrate the technology. Anno 2015 though, the playing field has changed. With more-and-more browsers supporting this new type of technology, the timing might be perfect to bring it to the masses.

In this thesis I will try to find if a WebGL based product configurator, is both viable in terms of technical and marketing aspects.
\newpage

%----------------------------------------------------------------------------------------
%   TOC
%----------------------------------------------------------------------------------------
\tableofcontents
\newpage


\section{Introduction}

Your introduction goes here! Some examples of commonly used commands and features are listed below, to help you get started.

If you have a question, please use the support box in the bottom right of the screen to get in touch. 

\section{Some \LaTeX{} Examples}
\label{sec:examples}

\subsection{Sections}

Use section and subsection commands to organize your document. \LaTeX{} handles all the formatting and numbering automatically. Use ref and label commands for cross-references.

\subsection{Comments}

Comments can be added to the margins of the document using the todo command, as shown in the example on the right. You can also add inline comments too:

\subsection{Tables and Figures}

Use the table and tabular commands for basic tables --- see Table~\ref{tab:widgets}, for example.

% Commands to include a figure:
%\begin{figure}
%\centering
%\includegraphics[width=0.5\textwidth]{frog.jpg}
%\caption{\label{fig:frog}This is a figure caption.}
%\end{figure}

\subsection{Mathematics}

\LaTeX{} is great at typesetting mathematics. Let $X_1, X_2, \ldots, X_n$ be a sequence of independent and identically distributed random variables with $\text{E}[X_i] = \mu$ and $\text{Var}[X_i] = \sigma^2 < \infty$, and let
$$S_n = \frac{X_1 + X_2 + \cdots + X_n}{n}
      = \frac{1}{n}\sum_{i}^{n} X_i$$
denote their mean. Then as $n$ approaches infinity, the random variables $\sqrt{n}(S_n - \mu)$ converge in distribution to a normal $\mathcal{N}(0, \sigma^2)$.

\subsection{Lists}

You can make lists with automatic numbering \dots

\begin{enumerate}
\item Like this,
\item and like this.
\end{enumerate}
\dots or bullet points \dots
\begin{itemize}
\item Like this,
\item and like this.
\end{itemize}

We hope you find write\LaTeX\ useful, and please let us know if you have any feedback using the help menu above.

\end{document}